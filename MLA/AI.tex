%% This is file `./samples/longsample.tex',
%% generated with the docstrip utility.
%%
%% The original source files were:
%%
%% apa7.dtx  (with options: `longsample')
%% ----------------------------------------------------------------------
%% 
%% apa7 - A LaTeX class for formatting documents in compliance with the
%% American Psychological Association's Publication Manual, 7th edition
%% 
%% Copyright (C) 2019 by Daniel A. Weiss <daniel.weiss.led at gmail.com>
%% 
%% This work may be distributed and/or modified under the
%% conditions of the LaTeX Project Public License (LPPL), either
%% version 1.3c of this license or (at your option) any later
%% version.  The latest version of this license is in the file:
%% 
%% http://www.latex-project.org/lppl.txt
%% 
%% Users may freely modify these files without permission, as long as the
%% copyright line and this statement are maintained intact.
%% 
%% This work is not endorsed by, affiliated with, or probably even known
%% by, the American Psychological Association.
%% 
%% ----------------------------------------------------------------------
%% 
\documentclass[man]{apa7}

\usepackage{lipsum}

\usepackage[american]{babel}

\usepackage{csquotes}
\usepackage[style=apa,sortcites=true,sorting=nyt,backend=biber]{biblatex}
\DeclareLanguageMapping{american}{american-apa}
\addbibresource{/workspaces/TexFile/MLA/bibliography.bib}

\title{Artificial Intelligence: Promise, Peril, and the Path Forward}
\shorttitle{AI: Promise, Peril, and Path Forward}

\author{Haolin Li\\Hayato Tonegawa\\Boyuan Zhao}

\affiliation{University of California\, Berkeley Extention}



\begin{document}
\maketitle


\section{Background}

\subsection{Understanding AI and Its History}
Artificial Intelligence (AI) involves machines and software mimicking human intelligence, capable of tasks such as learning, problem-solving, understanding natural language, and perception \parencite{russell2020}. The concept of AI dates back to the mid-20th century when British mathematician Alan Turing proposed the idea of machines simulating human intelligence. This proposal led to the Turing test, a significant milestone in AI history \parencite{turing1950computing}.

\subsection{The Current State and Popularity of AI}
The popularity and application of AI have grown exponentially in the last decade, transforming various sectors. Healthcare, education, automation, customer service - these are just a few areas where AI has shown potential in enhancing task efficiency and accuracy. In addition, technological advancements in big data and computational power have propelled AI's development \parencite{suthaharan2016}.

\section{The Positive Impact of AI}

\subsection{Individual and Societal Benefits}
AI has the potential to revolutionize individual productivity and overall societal development. Early AI implementations like Siri and Alexa simplified daily tasks, and today, more complex systems like ChatGPT understand and answer complex questions, increasing work efficiency \parencite{radford2019language}. Moreover, AI can connect to the internet, providing up-to-date information and more creative solutions that inspire new ideas and approaches.

\subsection{Technological Innovations}
Innovations in AI have provided us with tools that make our daily lives safer, more convenient, and enjoyable. Tesla's Autopilot system, for instance, has contributed to safer and easier driving, while the image-generating AI Midjourney allows even those without artistic skills to create beautiful images \parencite{tesla2021autopilot}.

\subsection{Industrial Revolution}
AI is poised to revolutionize many industries, leading to significant changes in traditional sectors. In healthcare, AI's ability to analyze large amounts of data can lead to more accurate diagnoses, personalized treatment plans, and improved patient outcomes. Education, too, can be transformed with AI-powered platforms catering to individual students' needs, fostering more effective and engaging educational experiences. The implications for the automotive industry, scientific research, and other sectors are equally profound \parencite{jiang2017}.

\subsection{Computational Power}
Computing power is the core of artificial intelligence, so why don't we use AI to accelerate computing? That’s exactly what NVIDIA is doing. NVIDIA's AI framework\parencite{nvidia2020ai} can greatly improve the computing power of hardware, making our personal computers run faster and making it easier for AI to be deployed at scale. Therefore, we are not far away from a high-tech inclusive society where AI is widely used. 

\subsection{Societal Impact}
AI's widespread implementation could lead to a more equitable and peaceful society by enhancing public safety, improving customer service, driving innovation, and promoting economic growth. The more efficient allocation of community resources could also help reduce societal disparities \parencite{brynjolfsson2019}.

\section{Potential Negative Impacts of AI}

Despite the numerous benefits of AI, there are also potential negative impacts that require consideration and mitigation:

1. \textbf{AI Bias}: AI algorithms can inherit human biases, leading to unintended discriminatory consequences. Mitigation strategies include responsible development and unbiased training \parencite{buolamwini2018}.

2. \textbf{Job Displacement}: AI could replace human roles in certain industries, leading to job losses. Proactive retraining programs and workforce support are crucial \parencite{arntz2016}.

3. \textbf{Shift in Human Experience}: Increased automation could reduce work time, necessitating new forms of meaningful activity.

4. \textbf{Global Regulations}: The international nature of AI technology demands harmonized global regulations for safe and effective interactions \parencite{schneider2020}.

5. \textbf{Accelerated Hacking}: AI could enable faster and more sophisticated cyber threats, requiring advanced cybersecurity measures \parencite{brundage2020}.

6. \textbf{AI Terrorism}: Advances in AI could lead to new forms of terrorism, necessitating robust security responses \parencite{brundage2020}.

\section{Addressing the Challenges of AI}

Addressing the challenges brought about by AI requires comprehensive solutions at different levels, from individual responsibility to collective and governmental measures.

\subsection{Bias Mitigation}
One of the first steps to addressing AI challenges is to mitigate the biases in AI systems. In the training phase, datasets must be diverse and comprehensive, reflecting the variety and nuances of real-world situations \parencite{gebru2018}. Moreover, these systems should be periodically tested and updated for any latent biases \parencite{mitchell2019}. Fairness in AI is a complex issue that requires an interdisciplinary approach, combining computer science, social science, and ethical considerations. Also, it's important to have diverse teams of AI developers, as their backgrounds and perspectives can inform the AI systems and help ensure they are built to understand and serve a broader range of people \parencite{holstein2019}.

\subsection{Job Displacement Solutions}
In addressing the threat of job displacement due to AI, it is essential to have robust retraining and upskilling programs to help the workforce adapt to the changes brought about by AI \parencite{arntz2016}. These programs should be designed to provide the skills needed for the jobs of the future and should be made accessible to workers of all backgrounds. Additionally, industries and government bodies should consider developing a social safety net for those who may lose their jobs due to AI-driven automation \parencite{chui2018}. 

\subsection{Regulatory Harmonization}
As AI technology is international in nature, the development of global regulations is critical for the safe and beneficial use of AI. International agreements and collaborations are necessary to harmonize these regulations, preventing misuse and fostering a climate of trust \parencite{schneider2020}. Governments, international bodies, and AI developers need to come together to build these regulations, ensuring that AI is developed and used responsibly.

\subsection{Cybersecurity and AI Terrorism}
AI technology may be exploited by hackers and malicious entities, leading to a need for advanced cybersecurity measures. Investment in cybersecurity technology and research must be made a priority, including AI-driven cybersecurity solutions \parencite{brundage2020}. Similarly, national and international security bodies must anticipate and prepare for potential AI-enabled forms of terrorism. This could involve early detection systems, effective countermeasures, and international collaborations on intelligence and security.

\subsection{Public Engagement and Transparency}
To avoid a shift in human experience that could result from the unchecked use of AI, it is essential to promote transparency and engage the public in discussions about AI development \parencite{bietti2020}. Public input can help ensure that AI development aligns with societal values and expectations, while transparency can enhance trust and acceptance of AI systems.

\subsection{Education and AI Literacy}
Given AI's growing impact on society, there is a need to foster AI literacy among the public \parencite{long2019}. Educational programs should incorporate AI-related content to prepare the younger generation for a future where AI will be ubiquitous. Similarly, adult learning programs can be utilized to increase AI understanding among the current workforce.

\printbibliography

\end{document}

%% 
%% Copyright (C) 2019 by Daniel A. Weiss <daniel.weiss.led at gmail.com>
%% 
%% This work may be distributed and/or modified under the
%% conditions of the LaTeX Project Public License (LPPL), either
%% version 1.3c of this license or (at your option) any later
%% version.  The latest version of this license is in the file:
%% 
%% http://www.latex-project.org/lppl.txt
%% 
%% Users may freely modify these files without permission, as long as the
%% copyright line and this statement are maintained intact.
%% 
%% This work is not endorsed by, affiliated with, or probably even known
%% by, the American Psychological Association.
%% 
%% This work is "maintained" (as per LPPL maintenance status) by
%% Daniel A. Weiss.
%% 
%% This work consists of the file  apa7.dtx
%% and the derived files           apa7.ins,
%%                                 apa7.cls,
%%                                 apa7.pdf,
%%                                 README,
%%                                 APA7american.txt,
%%                                 APA7british.txt,
%%                                 APA7dutch.txt,
%%                                 APA7english.txt,
%%                                 APA7german.txt,
%%                                 APA7ngerman.txt,
%%                                 APA7greek.txt,
%%                                 APA7czech.txt,
%%                                 APA7turkish.txt,
%%                                 APA7endfloat.cfg,
%%                                 Figure1.pdf,
%%                                 shortsample.tex,
%%                                 longsample.tex, and
%%                                 bibliography.bib.
%% 
%%
%% End of file `./samples/longsample.tex'.
