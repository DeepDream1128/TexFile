\documentclass[UTF8,titlepage,a4paper]{ctexart}
\usepackage{amsmath,amssymb,amsthm,amsfonts,amscd}
\usepackage{fontspec}
\setmainfont{Times New Roman}
\usepackage{graphicx}
\usepackage{titlesec}
\usepackage{makecell}
\usepackage{longtable}
\usepackage{xcolor}
\usepackage{tcolorbox}
\usepackage{soul}
\usepackage{adjustbox}
\usepackage{tcolorbox}
\usepackage{enumerate}
\usepackage{pdfpages}
\usepackage{float}
\usepackage{colortbl}
\usepackage{tabularx}
\usepackage{multirow}
\usepackage{pgfplots}
\usepackage{hyperref}
\numberwithin{figure}{section}
\usepackage[left=1.25in,right=1.25in,%
top=1in,bottom=1in]{geometry}
\usepackage{color}
\titleformat{\section}
  {\raggedright\LARGE\bfseries}{\thesection}{1em}{}
\begin{document}
\title{模电报告}
\author{赵伯远}
\date{\today}
\thispagestyle{empty}
\begin{center}
{\fontsize{30pt}{21pt}\selectfont \textbf{东华大学课程设计报告}}

\vspace{10cm}

\begin{tabular}{l}
    {\large 课程名称:电子技术设计与实践(模电)} \\
    \\
    \large{课题名称:\underline{\hspace{40pt}双极性全波精密整流电路\hspace{40pt}}}  \\
    \\
    \large{指导教师:\underline{\hspace{70pt}陈根龙\hspace{70pt}}} \\
    \\
    \large{学生姓名:\underline{\hspace{70pt}赵伯远\hspace{70pt}}} \\
    \\
    \large{学生班级学号:\underline{\hspace{11pt}人工智能2101 211440128\hspace{11pt}}} \\
    \end{tabular}
\end{center}
\clearpage
\setcounter{page}{1}
\tableofcontents
\clearpage
\section{摘要}
双极性全波精密整流电路是一种高效的整流电路设计,它能够将交流信号转换为直流信号,同时保留信号的双极性属性。这种电路设计利用运算放大器和二极管来实现精确的整流,确保在不同电源范围内的稳定性和效率。与传统的整流电路相比,双极性全波精密整流电路能够在较宽的电源范围内工作,使得各种输入信号可以进行全波整流\cite{ti}。此外,它通过减少二极管产生的压降,确保了输入与输出之间的高精度匹配,从而提高了整流效率。双极性全波精密整流电路的设计包括原理理解、组件选择、仿真测试和PCB设计,为实现高效和精确的整流提供了综合的解决方案\cite{ti}。通过现代仿真工具,如Multisim,可以进行电路性能的验证和优化,进一步确保了电路设计的准确性和可靠性\cite{csdn}。在多种应用中,双极性全波精密整流电路为实现高效、高精度的电力转换提供了重要的技术支持,展现出较高的实用价值和广泛的应用前景。

\section{设计任务}
\begin{enumerate}
    \item 利用基本的集成运算放大器、二极管和电阻等电子元件,设计并实现一个双向全波精密整流电路,以实现微弱交流信号的双向全波整流功能。
    \item 在Multisim软件平台上进行电路原理仿真,通过仿真结果对电路参数进行合理选取或调整,以确保电路的性能符合设计要求。
    \item 在便携式实验箱上构建并调试设计完成的电路,确保电路的实际性能与设计目标一致。
    \item 对比电路的实际输出、理论计算值和仿真结果,分析可能导致误差的因素,并提出可能的改进措施以优化电路性能。
\end{enumerate}

\section{设计指标}
\begin{enumerate}
    \item 电源电压范围:±12V
    \item 输入信号幅值范围:200mV至1V
    \item 输入信号波形:正弦波
    \item 输出评估:
       \begin{itemize}
           \item 输出波形的测量
           \item 输入输出信号幅值误差的测量与评估
       \end{itemize}
\end{enumerate}


\section{方案框图与简要原理}

\subsection{经典的精密整流电路}
\begin{itemize}
    \item \textbf{原理}:该设计利用运算放大器来补偿二极管的压降,实现精密整流。通常会有一个反相和非反相输入,以及一个二极管,它们连接到运算放大器的输出和输入,以确保在输入信号的正半周期和负半周期中实现整流。
    \item \textbf{优点}:电阻匹配简单,只需确保 \( R_1 = R2 \)。可以通过更改某些电阻的值来调节电路的增益。
\end{itemize}

\subsection{使用特定集成运算放大器(如LM358)的设计}
\begin{itemize}
    \item \textbf{原理}:LM358是一种低功耗、双运放的运算放大器,可以用于设计精密全波整流电路,以实现微弱交流信号的双向全波整流。
\end{itemize}

\subsection{优缺点比较}
\begin{itemize}
    \item \textbf{经典设计}的优点是简单和直接,但可能不适用于所有应用,尤其是在输入信号幅度变化较大或需要更高精度时。
    \item \textbf{使用特定集成运算放大器}的设计可能提供更好的性能和更高的精度,但可能需要更多的调试和优化。
\end{itemize}

\subsection{最终方案选择的理由}
在选择最终方案时,可能需要考虑以下因素:
\begin{itemize}
    \item \textbf{电路的简单性}:简单的电路可能更容易理解和实现。
    \item \textbf{所需的精度和性能}:如果需要更高的精度和性能,可能需要选择更复杂的设计或使用特定的集成运算放大器。
    \item \textbf{可调整性}:如果电路需要能够调整以适应不同的输入信号和应用,可能需要选择具有更多可调参数的设计。
\end{itemize}
\bibliographystyle{plain}
\bibliography{references}
\end{document}