\documentclass[tikz,border=10pt]{standalone}
\usepackage{tikz}
\usepackage{ctex}
\usetikzlibrary{shapes.geometric, arrows}

\tikzstyle{startstop} = [rectangle, rounded corners, minimum width=3cm, minimum height=1cm,text centered, draw=black, fill=red!30]
\tikzstyle{io} = [trapezium, trapezium left angle=70, trapezium right angle=110, minimum width=3cm, minimum height=1cm, text centered, draw=black, fill=blue!30]
\tikzstyle{process} = [rectangle, minimum width=3cm, minimum height=1cm, text centered, text width=3cm, draw=black, fill=orange!30]
\tikzstyle{decision} = [diamond, minimum width=3cm, minimum height=1cm, text centered, draw=black, fill=green!30]
\tikzstyle{arrow} = [thick,->,>=stealth]

\begin{document}
\begin{tikzpicture}[node distance=2cm]

\node (start) [startstop] {开始};
\node (in1) [io, below of=start] {输入学校数量、男子团体项目数量、女子团体项目数量};
\node (in2) [io, below of=in1] {输入学校名称};
\node (in3) [io, below of=in2] {输入项目名称、性别、名次取法};
\node (pr1) [process, below of=in3] {输入项目的各个名次的学校编号和得分};
\node (pr2) [process, below of=pr1] {计算每个学校的总分、男女团体总分};
\node (pr3) [process, below of=pr2] {排序学校};
\node (pr4) [process, below of=pr3] {查询};
\node (stop) [startstop, below of=pr4] {结束};

\draw [arrow] (start) -- (in1);
\draw [arrow] (in1) -- (in2);
\draw [arrow] (in2) -- (in3);
\draw [arrow] (in3) -- (pr1);
\draw [arrow] (pr1) -- (pr2);
\draw [arrow] (pr2) -- (pr3);
\draw [arrow] (pr3) -- (pr4);
\draw [arrow] (pr4) -- (stop);

\end{tikzpicture}
\end{document}
