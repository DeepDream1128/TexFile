\documentclass[12pt]{article}
\usepackage{amsmath,amssymb,amsthm,amsfonts,amscd,setspace}
\usepackage{fontspec}
\setmainfont{Times New Roman}
\usepackage{graphicx}
\usepackage{titlesec}
\usepackage{makecell}
\usepackage{longtable}
\usepackage{xcolor}
\usepackage{tcolorbox}
\usepackage{soul}
\usepackage{adjustbox}
\usepackage{tcolorbox}
\usepackage{enumerate}
\usepackage{pdfpages}
\usepackage{float}
\usepackage{colortbl}
\usepackage{tabularx}
\usepackage{multirow}
\usepackage{pgfplots}
\numberwithin{figure}{section}
\usepackage[left=1.25in,right=1.25in,%
top=1in,bottom=1in]{geometry}
\usepackage{color}
\renewcommand{\baselinestretch}{1}
\titleformat{\section}
  {\raggedright\LARGE\bfseries}{\thesection}{1em}{}
\begin{document}

\tableofcontents
\clearpage
\section{Background}
The history and development of Artificial Intelligence (AI) have revolutionized the digital landscape, fueling its rise to prominence in the contemporary era. The deep dive into AI's background, historical evolution, current development trends, and the reasons for its skyrocketing popularity elucidates our understanding of this game-changing technology.

\subsection{Artificial Intelligence: Origins and Evolution}
Artificial Intelligence (AI) is a specialized discipline within computer science that strives to empower machines with capabilities resembling human intelligence. This captivating and diverse field encompasses facets such as learning, reasoning, problem-solving, perception, and language comprehension.

The formal birth of AI as an academic discipline took place in 1956 at the Dartmouth Conference. Nonetheless, the conceptual underpinnings of AI predate this milestone. Historically, tales of artificial beings bestowed with intelligence have featured prominently in narratives, such as the automatons of Greek mythology.

In the early days of AI, researchers concentrated their efforts on devising symbolic methods. The goal was to program comprehensively all the knowledge needed by an intelligent entity into the machine. This phase in AI development, sometimes referred to as "Good Old-Fashioned AI" (GOFAI), led to significant strides in areas like expert systems and game-playing AI, most famously demonstrated by IBM's chess-playing computer, Deep Blue.

However, the enormity and complexity of capturing all of the world's knowledge in a machine-readable format became increasingly evident. This realization triggered a shift towards machine learning, which kicked off in earnest in the 1990s. Machine learning aimed to create systems capable of learning from data and improving from experience. The rise of the internet and the consequent explosion of available data provided the perfect breeding ground for these algorithms to evolve and flourish.

\subsection{Why AI's so Popular?}

\end{document}