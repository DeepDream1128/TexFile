\documentclass[12pt]{ctexart}
\usepackage{amsmath,amssymb,amsthm,amsfonts,amscd,setspace}
\usepackage{fontspec}
\setmainfont{Times New Roman}
\usepackage{graphicx}
\usepackage{titlesec}
\usepackage{makecell}
\usepackage{longtable}
\usepackage{xcolor}
\usepackage{tcolorbox}
\usepackage{soul}
\usepackage{adjustbox}
\usepackage{tcolorbox}
\usepackage{enumerate}
\usepackage{pdfpages}
\usepackage{float}
\usepackage{colortbl}
\usepackage{tabularx}
\usepackage{multirow}
\usepackage{pgfplots}
\numberwithin{figure}{section}
\usepackage[left=1.25in,right=1.25in,%
top=1in,bottom=1in]{geometry}
\usepackage{color}
\renewcommand{\baselinestretch}{1}
\titleformat{\section}
  {\raggedright\LARGE\bfseries}{\thesection}{1em}{}
\begin{document}

\tableofcontents
\clearpage
\section{Background}
\[T_{i}=\exp \left(-\sum_{j=1}^{i-1} \sigma_{j} \delta_{j}\right)\]

NeRF (Neural Radiance Fields) 中的 \(T_{i}\) 代表了从相机到射线上第 \(i\) 个点之前的累积“透射”(transmittance)。透射描述了光如何穿过介质,不被吸收或散射。

具体来说,公式
\[ T_{i} = \exp \left( -\sum_{j=1}^{i-1} \sigma_{j} \delta_{j} \right) \]
的定义可以这样解释:

- \(\sigma_{j}\) 是射线上第 \(j\) 个点的体密度(volume density)。一个较大的体密度意味着该点的材料更加不透明,光更容易被吸收或散射。
  
- \(\delta_{j}\) 是射线上两个相邻采样点之间的距离。它反映了光在介质中穿越的距离。
  
- \(\sum_{j=1}^{i-1} \sigma_{j} \delta_{j}\) 计算了从相机开始到达第 \(i\) 个点之前,光被吸收或散射的总“量”。这是一个累积的测量,它汇总了射线上所有先前点的效果。
  
- 使用 \(\exp(\cdot)\) 是为了模拟光的衰减效应。指数公式确保了透射始终处于 0 和 1 之间,其中 0 表示完全不透明(所有的光都被吸收或散射),1 表示完全透明(没有光被吸收或散射)。

为什么使用这个模型来表示透射呢?在真实世界中,光在穿过某种介质(如空气、水、雾等)时会按照其距离以指数方式衰减,这通常被称为“Beer-Lambert定律”。NeRF的这个公式基本上是对这个物理现象的一个简化和离散化的模拟。

总之,这个公式给出了一个在射线路径上某一点之前,光线遇到的累积吸收和散射的简化表示,从而帮助NeRF准确地模拟从场景的各个部分发射到相机的光。



这段公式描述了NeRF中的一个关键过程:volume rendering。在这里,我们将其分解以便更好地理解。

首先,让我们了解一些基本概念:

1. **体积渲染**: 这是一个计算方法,它通过模拟光线穿过半透明的物体来生成图像。在NeRF中,我们认为整个空间由很多微小的“体积元素”组成,每个体积元素都有一些颜色和密度。

2. **\(\boldsymbol{r}\)**: 这表示一个射线,例如从摄像机发出,穿越场景的射线。

3. **\(N\)**: 是射线\(\boldsymbol{r}\)上的采样点的数量。

4. **\(\boldsymbol{c}_{i}\)**: 表示第\(i\)个采样点的颜色。

5. **\(\sigma_{i}\)**: 表示第\(i\)个采样点的密度或“不透明度”。

6. **\(\delta_{i}\)**: 表示相邻采样点之间的距离。

现在让我们深入理解公式:

1. **\(w_{i}\)**:
   - 这表示第\(i\)个采样点的权重。

   - \(T_{i}\) 是从摄像机到第\(i\)个采样点之前所有采样点的“透明度”之积,即\(\prod_{j=1}^{i-1} \exp(-\sigma_j \delta_j)\)。这表示光线在到达第\(i\)个采样点之前所遭受的衰减。
   
   - \(\left(1-\exp \left(-\sigma_{i} \cdot \delta_{i}\right)\right)\) 是在第\(i\)个采样点上的“吸收”或“散射”量,描述了该点对结果颜色的贡献。
   
   - 因此,权重\(w_{i}\)是光线在第\(i\)个采样点上的衰减和吸收的结合。

2. **\(\hat{C}_{c}(\boldsymbol{r})\)**: 这是射线\(\boldsymbol{r}\)上所有采样点的加权颜色之和。每个采样点的颜色\(\boldsymbol{c}_{i}\)与其权重\(w_{i}\)相乘,并将所有结果求和。

3. **归一化**: 因为我们只是对一束射线进行采样和加权,归一化是确保最终的颜色值在合理范围内。通过将每个权重除以所有权重的总和,我们确保了权重的总和为1。

总体来说,这个公式描述了如何从一个射线上的多个采样点中获得一个合成的颜色值。每个采样点的贡献根据其密度和前面的所有采样点的透明度进行加权。
\end{document}